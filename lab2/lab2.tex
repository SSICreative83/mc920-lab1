\documentclass[10pt,a4paper]{article}
\usepackage[T1]{fontenc}
\usepackage[brazil]{babel}
\usepackage[utf8]{inputenc}


\usepackage{ae,aecompl}
\usepackage{pslatex}
\usepackage{epsfig}
\usepackage{geometry}
\usepackage{url}
\usepackage{textcomp}
\usepackage{ae}
\usepackage{subfig}
\usepackage{indentfirst}
\usepackage{textcomp}
\usepackage{color}
\usepackage{setspace}
\usepackage{verbatim}
\usepackage{mathtools}
\usepackage{amsmath}


\usepackage[compact]{titlesec}
\titlespacing{\section}{0pt}{*0}{*0}
\titlespacing{\subsection}{0pt}{*0}{*0}
\titlespacing{\subsubsection}{0pt}{*0}{*0}

\linespread{1.5}
\geometry{ 
  a4paper,	% Formato do papel
  tmargin=25mm,	% Margem superior
  bmargin=25mm,	% Margem inferior
  lmargin=20mm,	% Margem esquerda
  rmargin=20mm,	% Margem direita
  footskip=10mm	% Espaço entre o rodapé e o fim do texto
}
%  ABACO -- Conjunto de macros para desenhar o 'abaco

%  Desenho original de Hans Liesenberg

%  Macros de Tomasz Kowaltowski

%  DCC -- IMECC -- UNICAMP

%  Mar,co de 1988  --  Vers~ao 1.0

% Ajustado para LaTeX da SUN -- Mar,co de 1991

% ---------------------------------------------------------

%  Chamada:   \ABACO{d1}{d2}{d3}{d4}{esc}
%             com:  di's -- os quatro d'igitos;
%	           esc  -- fator de escala

% ---------------------------------------------------------

%  DEFINI,C~OES AUXILIARES

% ---------------------------------------------------------


%  Forma o d'igito pequeno (0 ou 1)

\newcommand{\ABACODP}[1]{%
%
\thicklines
%    
\begin{picture}(8,0)
    \ifcase#1{   %  caso 0
       \put(0,0)    {\line(1,0){4}}
       \multiput(5,0)(2,0){2}{\oval(2,4)}}
    \or{         %  caso 1
       \put(2,0)    {\line(1,0){4}}
       \multiput(1,0)(6,0){2}{\oval(2,4)}}
    \fi
\end{picture}
    } % \ABACODP

% Forma o d'igito grande (0 a 4)

\newcommand{\ABACODG}[1]{%
%
\thicklines
%    
\begin{picture}(14,0)
    \ifcase#1{   % caso 0
       \multiput(1,0)(2,0){5}{\oval(2,4)}}
       \put(10,0)   {\line(1,0){4}}
    \or{         % caso 1
       \multiput(1,0)(2,0){4}{\oval(2,4)}}
       \put(8,0)   {\line(1,0){4}}
       \put(13,0)   {\oval(2,4)}
    \or{         % caso 2
       \multiput(1,0)(2,0){3}{\oval(2,4)}
       \put(6,0)   {\line(1,0){4}}
       \multiput(11,0)(2,0){2}{\oval(2,4)}}
    \or{         % caso 3
       \multiput(1,0)(2,0){2}{\oval(2,4)}
       \put(4,0)   {\line(1,0){4}}
       \multiput(9,0)(2,0){3}{\oval(2,4)}}
    \or{         % caso 4
       \put(1,0)  {\oval(2,4)}}
       \put(2,0)   {\line(1,0){4}}
       \multiput(7,0)(2,0){4}{\oval(2,4)}
    \fi
\end{picture}
    } % \ABACODG
       
% Forma um d'igito (0 a 9)

\newcommand{\ABACOD}[1]{%
%
    \ifnum#1>9
       \errmessage{#1: Argumento invalido para ABACO}
    \fi
    \ifnum#1<0
       \errmessage{#1: Argumento invalido para ABACO}
    \fi
%
\begin{picture}(24,0)
%    
    \ifnum#1<5
       \put(16,0) {\ABACODP{0}}
    \else   
       \put(16,0) {\ABACODP{1}}
    \fi
%    
    \ifnum#1<5
       \put(0,0)  {\ABACODG{#1}}
    \else
       \ifcase#1\or \or \or \or
          \or  \put(0,0)  {\ABACODG{0}}
          \or  \put(0,0)  {\ABACODG{1}}
          \or  \put(0,0)  {\ABACODG{2}}
          \or  \put(0,0)  {\ABACODG{3}}
          \or  \put(0,0)  {\ABACODG{4}}
       \fi
    \fi   
\end{picture}
    } % \ABACOD
    
% -------------------------------------------------

%  DEFINI,C~AO PRINCIPAL
    
\newcommand{\ABACO}[5]{%
    \setlength{\unitlength}{#5mm}
%
    \thinlines
%   
\begin{picture}(28,25)
%   
% moldura
%
% externa
%
        \put(0,0)            {\line(0,1){25}}
        \put(0,0)            {\line(1,0){28}}
        \put(28,0)           {\line(0,1){25}}
        \put(0,25)           {\line(1,0){28}}
% interna
        \put(2,2)            {\line(0,1){21}}
	\put(26,2)           {\line(0,1){21}}
	\put(16,2)           {\line(0,1){21}}
	\put(18,2)           {\line(0,1){21}}
	\put(2,2)            {\line(1,0){14}}
	\put(16,2)           {\line(1,-1){1}}
	\put(17,1)           {\line(1,1){1}}
	\put(18,2)           {\line(1,0){8}}
	\put(2,23)           {\line(1,0){14}}
	\put(16,23)          {\line(1,1){1}}
	\put(17,24)          {\line(1,-1){1}}
	\put(18,23)          {\line(1,0){8}}
	\put(0,0)            {\line(1,1){2}}
	\put(0,25)           {\line(1,-1){2}}
	\put(28,0)           {\line(-1,1){2}}
	\put(28,25)          {\line(-1,-1){2}}
%
%   
% d'igitos
%
%   
       \put(2,20)  {\ABACOD{#1}}
       \put(2,15)  {\ABACOD{#2}}
       \put(2,10)  {\ABACOD{#3}}
       \put(2,5)   {\ABACOD{#4}}
%      
\end{picture}
    } % \ABACO
    
 
\renewcommand{\thetable}{\Roman{table}}
\newcommand{\x} {$\bullet$}


\begin{document}
% CAPA
\begin{titlepage}
  \thispagestyle{empty}
  \begin{center} {\large \textbf{UNIVERSIDADE~ESTADUAL~DE~CAMPINAS}} \end{center}
  \begin{center} {\large INSTITUTO~DE~COMPUTAÇÃO}                    \end{center}
  \vspace{0.1cm}
  \begin{center}
    \begin{minipage}[tl]{31mm}
      \ABACO{1}{9}{6}{9}{1}
    \end{minipage}
  \end{center}
  \vspace{0.3cm}
  \begin{center} 
    {\large \textsc{"Detecção de padrões de legendas em imagens de ritmo visual a
partir do detector de Harris"?  }} 
    \\\vspace{0.5cm}
    {\textsl{Relatório do segundo de MC920}}
    \\\vspace{1cm}
    \begin{tabular}{rl}
      \textbf{Aluno}:   Carlos~Eduardo~Rosa~Machado &
      \textbf{RA}:          059582 \\ 
      \textbf{Aluno}:        Tiago~Chedraoui~Silva & 
      \textbf{RA}:        082941 \\
      \textbf{Aluno}:        William~Marques~Dias & 
      \textbf{RA}:        065106 \\
    \end{tabular}
  \end{center}
  \vspace{0.5cm}


  \begin{abstract}
A consistência de um filto de bordas é de suma importância para
interpretações de sequências de imagens 3D para recursos que utilizam
algoritmos rastreamento. 

Para abranger as regiões da iagem que contém testura e características isoladas, 
uma combinação de detector de bordas e cantos baseados na função de
auto-correlação local é utilizado.

A partir do dector de Harris, avaliou-se a detecção de padrões de
legendas em imagens de ritmo visual.

  \end{abstract}
  % Sumário
  \tableofcontents
\end{titlepage} 

\vspace{2mm}
\newpage

\section{Introdução}



\section{Métodos}

%corner = canto O.o
Desenvolveu-se em python~\cite{python} um programa para aplicar o
detector de cantos de Moravec.

Dado uma imagem $I$ retorna-se a imagem com os cantos realçados.
Para isso aplica-se a fórmula:

\begin{equation}
E_{x,y}=\sum_{u,v}w_{u,v}\left | I_{x+u,y+v}-I_{u,v}  \right |^2
\end{equation}

No entanto, o operador de Moravec sofre de aulgums probemas cujas soluções são apresentados no paper de  Chris Hafris e
Mikes Stephens~\cite{paper}. Abaixo estão listadas as que serviram de
base para uma implementação  em
nossa pesquisa. 

\begin{enumerate}
\item \textbf{A resposta é anisotrópica, porque somente um
conjunto discreto de deslocamentos a cada 45 graus é
considerado} - Todos os pequenos deslocamentos são cobertos
realizando uma expansão analítica sobre a origem do deslocamento.
Assim:

\begin{equation}
E_{x,y}=\sum_{u,v}w_{u,v}\left | xX +yY + O(x^2,x^2) \right  |^2 
\end{equation}

Em que:
\begin{eqnarray*}
X = 1 \otimes (-1,0,1)= \delta I \delta x\\
Y = 1 \otimes (-1,0,1)^T= \delta I \delta y
\end{eqnarray*}

Para pequenos deslocamentos, E pode ser escrito como:

\begin{equation}
E_{x,y}=Ax^2+ 2Cxy + By^2 
\end{equation}

Em que 
\begin{eqnarray*}
A= X^2\otimes w\\
B= Y^2\otimes w\\
C= (XY)\otimes w
\end{eqnarray*}

\item \textbf{A resposta é ruidosa devido ao fato de a  janela ser
    binária e retangular} - Usar uma janela suave circular, como por
  exemplo uma Gaussiana.

\begin{equation}
w_{x,y}=\exp-(u^2+ v^2)/2\sigma^2 
\end{equation}

\item \textbf{O operador responde muito rapidamente à bordas 
porque somente o mínimo de E é levado em
conta} - reformular a medida do canto para fazer uso da variação de E com a direção da mudança.

A mudança, E para o pequeno deslocamento (x, y) pode ser escrita como:

\begin{equation}
E_{x,y}=(x,y)M(x,y)^T
\end{equation}

Emque a Matrix simétrica 2x2 é dada por:


\[
 M = \begin{bmatrix}
       A & C \\[0.3em]
       C & B \\[0.3em]
     \end{bmatrix}
\]

Usando a matrix M calculamos:
\begin{eqnarray*}
Tr(M) = A + B\\
Det(M) = AB-C^2
\end{eqnarray*}

Para realizar um avaliar os cantos, fazemos:
\begin{equation}
R = Det - kTr^2
\end{equation}

Em que se $R>0$, temos uma região de canto, se $R<0$ é uma região e
borda, se $R \approx 0$ temos uma região plana.
\end{enumerate}

\section{Comparação de imagens}
Aplicando as melhorias propostas no paper de Harris e Stephens,
fez-se uma sequência de experimentos.

\section{Resultados}

\section{Conclusão}
% Necessária?
% \section{Agradecimentos}

% ******************************************************
% REFERENCIAS BIBLIOGRÁFICAS
% ******************************************************
% \section{Referências}
\bibliographystyle{plain}
\begin{small}
  \bibliography{referencias}
\end{small}

\end{document}
